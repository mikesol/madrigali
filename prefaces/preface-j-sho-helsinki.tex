\documentclass[a4paper]{book}
%\usepackage{geometry}                % See geometry.pdf to learn the layout options. There are lots.
%\usepackage[paperwidth=11in,paperheight=17in,margin=1.00in]{geometry}
%\usepackage[a3paper]{geometry} 
%\usepackage[paperwidth=297mm,paperheight=420mm,margin=25mm]{geometry}
%\geometry{landscape}                % Activate for for rotated page geometry
%\usepackage[parfill]{parskip}    % Activate to begin paragraphs with an empty line rather than an indent
\usepackage{graphicx}
\usepackage{newcent}
\usepackage{amssymb}
\usepackage{float}
\usepackage{verbatim}
\usepackage{hyperref}
\usepackage{multirow}
\begin{document}
\begin{titlepage}
\begin{flushright}
\vspace*{\stretch{1}}
{\Huge \bfseries j-sho \\ }
{\large for four-part choir \\}
%{\large for bass flute, bass clarinet, tuba doubling on piano, harp, soprano, and percussion \\}
%{\large \bfseries for bass flute, bass clarinet, tuba/piano, soprano, harp, percussion, and tape \\}
\par
\vspace{\stretch{1}}
%{\Large Date of composition: 06/09-08/09}
{\Large Mike Solomon}
%{\large Marge de Couture (motto di riconoscimento) \\}
%{\large BMI Student Composer Awards}
\end{flushright}
\end{titlepage}
\thispagestyle{empty} 
\frontmatter
%\fontsize{14}{17}
%\vspace*{\stretch{1}} \begin{center}
%\emph{this work is dedicated Paul Koonce, who taught me how to write for wind ensemble}
%\end{center}
%\vspace*{\stretch{3}} 
\clearpage
%\vspace*{\stretch{3}}
\emph{j-sho} takes its inspiration from my numerous visits to the Jersey
shore as well as my fascination with the work of Ryan Trecartin (see, for
example, \url{http://vimeo.com/5841178}).  It uses
the same convention as Berio in his Sequenza
(\url{http://www.youtube.com/watch?v=DGovCafPQAE})
to denote sung (5-line staff) versus spoken (1-line staff) versus somewhere
between the two (3-line staff, with the middle indicating the middle of the
singer's tessitura).  The work calls for various percussive
effects, the most important of which being the vacuum cleaner.  The others
can be modified to fit the availability of the ensemble.  For example, bass
drum can be a low-sounding drum, gong a cymbal, etc..
\vspace{\stretch{2}}
\end{document}